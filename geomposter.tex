\documentclass[17pt, a0paper, landscape]{tikzposter} % See Section 3
\tikzposterlatexaffectionproofoff % отключение вотермарки

\usepackage[T2A]{fontenc}
\usepackage[utf8]{inputenc}
\usepackage{polyglossia}
\setmainlanguage{russian}
\setotherlanguage{english}
\setkeys{russian}{babelshorthands=true}

\usepackage{amsmath,amssymb}
\usepackage[shortlabels]{enumitem}
\usepackage{graphicx}
\usepackage{comment}
\usepackage{parskip}
\usepackage[unicode]{hyperref}
\AddEnumerateCounter{\asbuk}{\russian@asbuk@alph}{}
\usepackage{pgfplots}
\pgfplotsset{compat=1.15}
\usepackage{mathrsfs}
\usetikzlibrary{arrows}

% команда \bulurl для создания голубой ссылки
\usepackage{xcolor}
\usepackage[normalem]{ulem}
\useunder{\uline}{\ulined}{}%
\DeclareUrlCommand{\bulurl}{\def\UrlFont{\ttfamily\color{blue}\ulined}}

% changing font
\usepackage{PTSans}

\setmainfont{PT Sans}
\setromanfont{PT Sans} 
\setsansfont{PT Sans} 
\setmonofont{Consolas} 

\newfontfamily{\cyrillicfont}{PT Sans} 
\newfontfamily{\cyrillicfontrm}{PT Sans}
\newfontfamily{\cyrillicfonttt}{PT Sans}
\newfontfamily{\cyrillicfontsf}{PT Sans}

\usepackage{unicode-math}
\setmathfont{Fira Math}

\definecolorpalette{PurpleGrayBlue}{
    \definecolor{colorOne}{HTML}{D40279}
    \definecolor{colorTwo}{HTML}{7F8897}
    \definecolor{colorThree}{HTML}{006C9E}
}

\settitle{\centering \@title}
\title{\textcolor[HTML]{D40279}{\textbf{\Huge{Подготовка к ЕГЭ по математике: планиметрия}}}}  % See Section 4.1

\makeatletter
\newcommand\insertlogoi[2][]{\def\@insertlogoi{\includegraphics[#1]{#2}}}
\newcommand\insertlogoii[2][]{\def\@insertlogoii{\includegraphics[#1]{#2}}}
\newlength\LogoSep
\setlength\LogoSep{0pt}

\insertlogoi[width=10cm]{figures/logo.pdf}
\insertlogoii[width=5cm]{example-image-b}

\renewcommand\maketitle[1][]{  % #1 keys
    \normalsize
    \setkeys{title}{#1}
    % Title dummy to get title height
    \node[transparent,inner sep=\TP@titleinnersep, line width=\TP@titlelinewidth, anchor=north, minimum width=\TP@visibletextwidth-2\TP@titleinnersep]
        (TP@title) at ($(0, 0.5\textheight-\TP@titletotopverticalspace)$) {\parbox{\TP@titlewidth-2\TP@titleinnersep}{\TP@maketitle}};
    \draw let \p1 = ($(TP@title.north)-(TP@title.south)$) in node {
        \setlength{\TP@titleheight}{\y1}
        \setlength{\titleheight}{\y1}
        \global\TP@titleheight=\TP@titleheight
        \global\titleheight=\titleheight
    };

    % Compute title position
    \setlength{\titleposleft}{-0.5\titlewidth}
    \setlength{\titleposright}{\titleposleft+\titlewidth}
    \setlength{\titlepostop}{0.5\textheight-\TP@titletotopverticalspace}
    \setlength{\titleposbottom}{\titlepostop-\titleheight}

    % Title style (background)
    \TP@titlestyle

    % Title node
    \node[inner sep=\TP@titleinnersep, line width=\TP@titlelinewidth, anchor=north, minimum width=\TP@visibletextwidth-2\TP@titleinnersep]
        at (0,0.5\textheight-\TP@titletotopverticalspace)
        (title)
        {\parbox{\TP@titlewidth-2\TP@titleinnersep}{\TP@maketitle}};

    \node[inner sep=400pt,anchor=west] 
      at ([xshift=-\LogoSep]title.west)
      {\@insertlogoi};

    \node[inner sep=400pt,anchor=east] 
      at ([xshift=\LogoSep]title.east)
      {\Huge\bulurl{https://plucik.ru/}};

    % Settings for blocks
    \normalsize
    \setlength{\TP@blocktop}{\titleposbottom-\TP@titletoblockverticalspace}
}
\makeatother
\usetheme{Simple} % See Section 5
\colorlet{titlebgcolor}{white}
\usetitlestyle{Default}
\useblockstyle{TornOut}

\newcommand{\parallelsum}{\mathbin{\|}}

\begin{comment} Этих блоков нету
\block{Отношение площадей подобных фигур}
{
$$k \text{ - коэффициент подобия, } \quad \frac {S_1} {S_2} = k^2$$
\begin{center}
\input{figures/area_similar_plain.tex}
\end{center}
}
\end{comment}

\begin{document}

\definecolor{atfczz}{rgb}{0,0.6,0}
\definecolor{fruycc}{rgb}{0.8313725490196079,0.00784313725490196,0.4745098039215686}
\definecolor{qqwuqq}{rgb}{0,0.6,0}
\definecolor{ududff}{rgb}{0.8313725490196079,0.00784313725490196,0.4745098039215686}
\definecolor{qqqqff}{rgb}{0.8313725490196079,0.00784313725490196,0.4745098039215686}
\definecolor{atfcqq}{rgb}{0,0.6,0}
\definecolor{xdxdff}{rgb}{0.8313725490196079,0.00784313725490196,0.4745098039215686}
\definecolor{duqsxz}{rgb}{0.8313725490196079,0.00784313725490196,0.4745098039215686}

\maketitle

\begin{columns}
\column{0.166}
\block{Теорема Фалеса}
{
$$l_1 \ \parallelsum \ l_2 \ \parallelsum \ l_3 \iff \frac {a} {a'} = \frac {b} {b'} = \frac {c} {c'}$$
\begin{center}
\begin{tikzpicture}[line cap=round,line join=round,>=triangle 45,x=1cm,y=1cm, every node/.style={scale=2}]
\draw [line width=2pt,color=atfczz] (4.195088676671214,2.0236729844379235)-- (5.16643929058663,0.028998659625471466);
\draw [line width=2pt] (0,0)-- (5.16643929058663,0.028998659625471466);
\draw [line width=2pt] (5.16643929058663,0.028998659625471466)-- (9.383603174261191,0.052669140041321685);
\draw [line width=2pt] (9.383603174261191,0.052669140041321685)-- (12.579502562664695,0.07060737435492814);
\draw [line width=2pt] (12.579502562664695,0.07060737435492814)-- (13.950433036654262,0.0783022574164294);
\draw [line width=2pt] (0,0)-- (4.195088676671214,2.0236729844379235);
\draw [line width=2pt] (4.195088676671214,2.0236729844379235)-- (7.619376752271763,3.6755187029947947);
\draw [line width=2pt] (7.619376752271763,3.6755187029947947)-- (10.214409923473449,4.927339326354816);
\draw [line width=2pt] (10.214409923473449,4.927339326354816)-- (11.716180041642744,5.651779702043574);
\draw [line width=2pt,color=atfczz] (10.214409923473449,4.927339326354816)-- (12.579502562664695,0.07060737435492814);
\draw [line width=2pt,color=atfczz] (7.619376752271763,3.6755187029947947)-- (9.383603174261191,0.052669140041321685);
\begin{scriptsize}
\draw [fill=qqqqff] (0,0) circle (2.5pt);
\draw [fill=qqqqff] (5.16643929058663,0.028998659625471466) circle (2.5pt);
\draw [fill=qqqqff] (4.195088676671214,2.0236729844379235) circle (2.5pt);
\draw[color=atfczz] (5.133697135061391,1.275670112633254) node {$l_1$};
\draw [fill=qqqqff] (7.619376752271763,3.6755187029947947) circle (2.5pt);
\draw [fill=qqqqff] (10.214409923473449,4.927339326354816) circle (2.5pt);
\draw [fill=qqqqff] (9.383603174261191,0.052669140041321685) circle (2.5pt);
\draw [fill=qqqqff] (12.579502562664695,0.07060737435492814) circle (2.5pt);
\draw[color=black] (2.6343792633015006,-0.4) node {$a'$};
\draw[color=black] (7.425648021828104,-0.4) node {$b'$};
\draw[color=black] (11.10368349249659,-0.4) node {$c'$};
\draw[color=black] (1.8158253751705322,1.366696917138584) node {$a$};
\draw[color=black] (5.679399727148704,3.1951483815499984) node {$b$};
\draw[color=black] (8.768076398362892,4.675323376549715) node {$c$};
\draw[color=atfczz] (11.824010914051842,2.8270834763762727) node {$l_3$};
\draw[color=atfczz] (8.866302864938609,2.146361286162499) node {$l_2$};
\end{scriptsize}
\end{tikzpicture}
\end{center}
}
\block{Точка пересечения медиан треугольника}
{
$$\frac {CO} {OM} = \frac {AO} {OD} = \frac {BO} {OE} = \frac {2} {1}$$
\begin{center}
\begin{tikzpicture}[line cap=round,line join=round,>=triangle 45,x=1cm,y=1cm, every node/.style={scale=2}]
\draw [line width=2pt] (4,7)-- (2,3.5);
\draw [line width=2pt] (3.0980806838901738,5.266539596203184) -- (2.96654613845051,5.346557342893013);
\draw [line width=2pt] (3.0657672727198317,5.209991126655085) -- (2.9342327272801683,5.290008873344914);
\draw [line width=2pt] (3.0334538615494897,5.153442657106986) -- (2.9019193161098262,5.2334604037968155);
\draw [line width=2pt] (2,3.5)-- (0,0);
\draw [line width=2pt] (1.0980806838901735,1.7665395962031814) -- (0.9665461384505106,1.8465573428930124);
\draw [line width=2pt] (1.0657672727198317,1.7099911266550836) -- (0.9342327272801687,1.7900088733449147);
\draw [line width=2pt] (1.0334538615494897,1.6534426571069858) -- (0.9019193161098268,1.7334604037968169);
\draw [line width=2pt] (0,0)-- (6,0);
\draw [line width=2pt] (2.9681885994494936,0.07877461706783374) -- (2.9681885994494936,-0.07877461706783374);
\draw [line width=2pt] (3.0318114005505064,0.07877461706783374) -- (3.0318114005505064,-0.07877461706783374);
\draw [line width=2pt] (6,0)-- (12,0);
\draw [line width=2pt] (8.968188599449494,0.07877461706783374) -- (8.968188599449494,-0.07877461706783374);
\draw [line width=2pt] (9.031811400550506,0.07877461706783374) -- (9.031811400550506,-0.07877461706783374);
\draw [line width=2pt] (12,0)-- (8,3.5);
\draw [line width=2pt] (9.950620077642903,1.689920420061111) -- (10.0493799223571,1.8100795799388871);
\draw [line width=2pt] (8,3.5)-- (4,7);
\draw [line width=2pt] (5.950620077642902,5.189920420061112) -- (6.049379922357097,5.310079579938887);
\draw [line width=2pt] (2,3.5)-- (5.333333333333333,2.3333333333333335);
\draw [line width=2pt] (5.333333333333333,2.3333333333333335)-- (12,0);
\draw [line width=2pt] (5.333333333333333,2.3333333333333335)-- (8,3.5);
\draw [line width=2pt] (5.333333333333333,2.3333333333333335)-- (0,0);
\draw [line width=2pt] (5.333333333333333,2.3333333333333335)-- (6,0);
\draw [line width=2pt] (5.333333333333333,2.3333333333333335)-- (4,7);
\begin{scriptsize}
\draw [fill=qqqqff] (0,0) circle (2.5pt);
\draw[color=qqqqff] (-0.497392325492684,-0.22538293216630612) node {$A$};
\draw [fill=qqqqff] (12,0) circle (2.5pt);
\draw[color=qqqqff] (12.532557339994788,-0.19912472647702817) node {$B$};
\draw [fill=qqqqff] (4,7) circle (2.5pt);
\draw[color=qqqqff] (3.828958149376204,7.770240700218819) node {$C$};
\draw [fill=qqqqff] (6,0) circle (2pt);
\draw[color=qqqqff] (5.941235145929837,-0.35667396061269574) node {$M$};
\draw [fill=qqqqff] (8,3.5) circle (2pt);
\draw[color=qqqqff] (8.257105106006712,3.9234135667396046) node {$D$};
\draw [fill=qqqqff] (2,3.5) circle (2pt);
\draw[color=qqqqff] (1.5639864301801387,3.8577680525164095) node {$E$};
\draw [fill=qqqqff] (5.333333333333333,2.3333333333333335) circle (2pt);
\draw[color=qqqqff] (5.075965050956059,1.7308533916848985) node {$O$};
\end{scriptsize}
\end{tikzpicture}
\end{center}
}
\block{Замечательное свойство трапеции}
{
\begin{center}
$F, M$ - середины оснований трапеции $ABCD$\\
$\Rightarrow$ точки $E, F, M$ лежат на одной прямой \\
\begin{tikzpicture}[line cap=round,line join=round,>=triangle 45,x=1cm,y=1cm, every node/.style={scale=2}]
\draw [line width=2pt] (0,0)-- (6,0);
\draw [line width=2pt] (3,0.0787746170678337) -- (3,-0.0787746170678337);
\draw [line width=2pt] (6,0)-- (12,0);
\draw [line width=2pt] (9,0.0787746170678337) -- (9,-0.0787746170678337);
\draw [line width=2pt] (7,6)-- (6.616071576888137,3.696429461328818);
\draw [line width=2pt] (6.616071576888137,3.696429461328818)-- (6,0);
\draw [line width=2pt] (4.312501038216955,3.6964294613288184)-- (6.616071576888137,3.696429461328818);
\draw [line width=2pt] (5.4349557347437125,3.775204078396653) -- (5.4349557347437125,3.617654844260986);
\draw [line width=2pt] (5.49361688036138,3.775204078396653) -- (5.49361688036138,3.617654844260986);
\draw [line width=2pt] (6.616071576888137,3.696429461328818)-- (8.919642115559318,3.696429461328818);
\draw [line width=2pt] (7.7385262734148945,3.775204078396652) -- (7.7385262734148945,3.6176548442609846);
\draw [line width=2pt] (7.797187419032562,3.775204078396652) -- (7.797187419032562,3.6176548442609846);
\draw [line width=2pt] (7,6)-- (4.312501038216955,3.6964294613288184);
\draw [line width=2pt] (4.312501038216955,3.6964294613288184)-- (0,0);
\draw [line width=2pt] (7,6)-- (8.919642115559318,3.696429461328818);
\draw [line width=2pt] (8.919642115559318,3.696429461328818)-- (12,0);
\begin{scriptsize}
\draw [fill=qqqqff] (0,0) circle (2.5pt);
\draw[color=qqqqff] (-0.4368530020703934,-0.27789934354485735) node {$A$};
\draw [fill=qqqqff] (12,0) circle (2.5pt);
\draw[color=qqqqff] (12.35127674258109,-0.1728665207877457) node {$D$};
\draw [fill=atfczz] (7,6) circle (2.5pt);
\draw[color=atfczz] (6.391304347826087,6.155361050328228) node {$E$};
\draw [fill=qqqqff] (4.312501038216955,3.6964294613288184) circle (2.5pt);
\draw[color=qqqqff] (3.751552795031056,3.923413566739607) node {$B$};
\draw [fill=atfczz] (6,0) circle (2.5pt);
\draw[color=atfczz] (5.957211870255349,-0.40919037199124675) node {$M$};
\draw [fill=qqqqff] (8.919642115559318,3.696429461328818) circle (2.5pt);
\draw[color=qqqqff] (9.300897170462388,3.9102844638949676) node {$C$};
\draw [fill=atfczz] (6.616071576888137,3.696429461328818) circle (2.5pt);
\draw[color=atfczz] (6.109730848861284,3.266958424507659) node {$F$};
\end{scriptsize}
\end{tikzpicture}
\end{center}
}
\block{Теорема Пифагора}
{
$$\Delta ABC \text{ - прямоугольный } \iff a^2 + b^2 = c^2$$
\begin{center}
\begin{tikzpicture}[line cap=round,line join=round,>=triangle 45,x=1cm,y=1cm, every node/.style={scale=2}]
\draw[line width=2pt,color=atfczz,fill=atfczz,fill opacity=0.1] (3.9379162856424044,5.272150430183971) -- (4.419105231754571,4.912736977844356) -- (4.778518684094185,5.393925923956522) -- (4.297329737982019,5.753339376296137) -- cycle; 
\draw [line width=2pt] (0,0)-- (12,0);
\draw [line width=2pt] (12,0)-- (4.297329737982019,5.753339376296137);
\draw [line width=2pt] (4.297329737982019,5.753339376296137)-- (0,0);
\begin{scriptsize}
\draw [fill=qqqqff] (0,0) circle (2.5pt);
\draw[color=qqqqff] (-0.8105475204138632,-0.4469140551523894) node {$A$};
\draw [fill=qqqqff] (12,0) circle (2.5pt);
\draw[color=qqqqff] (12.72286286388731,-0.33366375904945084) node {$B$};
\draw [fill=qqqqff] (4.297329737982019,5.753339376296137) circle (2.5pt);
\draw[color=qqqqff] (4.115840360063969,6.4) node {$C$};
\draw[color=black] (5.984470245762457,-0.36197633307518545) node {$c$};
\draw[color=black] (8.84404022236166,3.6867217526048695) node {$b$};
\draw[color=black] (1.539396123722114,3.658409178579135) node {$a$};
\end{scriptsize}
\end{tikzpicture}
\end{center}
}
\block{Высота прямоугольного треугольника}
{
$$h^2 = x \cdot y$$
\begin{center}
\begin{tikzpicture}[line cap=round,line join=round,>=triangle 45,x=1cm,y=1cm, every node/.style={scale=2}]
\draw[line width=2pt,color=atfczz,fill=atfczz,fill opacity=0.1] (8.142420844509443,0) -- (8.142420844509443,0.4242640687119288) -- (7.718156775797515,0.42426406871192873) -- (7.718156775797515,0) -- cycle; 
\draw[line width=2pt,color=atfczz,fill=atfczz,fill opacity=0.1] (7.377903411909716,5.495301795242367) -- (7.631335154915672,5.155048431354568) -- (7.971588518803472,5.408480174360525) -- (7.718156775797515,5.748733538248324) -- cycle; 
\draw [line width=2pt] (0,0)-- (7.718156775797515,5.748733538248324);
\draw [line width=2pt] (7.718156775797515,5.748733538248324)-- (12,0);
\draw [line width=2pt] (7.718156775797515,5.748733538248324)-- (7.718156775797515,0);
\draw [line width=2pt] (0,0)-- (7.718156775797515,0);
\draw [line width=2pt] (7.718156775797515,0)-- (12,0);
\begin{scriptsize}
\draw [fill=fruycc] (0,0) circle (2.5pt);
\draw [fill=fruycc] (12,0) circle (2.5pt);
\draw [fill=fruycc] (7.718156775797515,5.748733538248324) circle (2.5pt);
\draw [fill=fruycc] (7.718156775797515,0) circle (2.5pt);
\draw[color=black] (7.22,2.861818181818186) node {$h$};
\draw[color=black] (3.8,-0.6181818181818155) node {$x$};
\draw[color=black] (9.96,-0.5381818181818155) node {$y$};
\end{scriptsize}
\end{tikzpicture}
\end{center}
}

\column{0.166}
\block{Теорема синусов}
{
$$\frac {a} {\sin{\alpha}} = \frac {b} {\sin{\beta}} = \frac {c} {\sin{\gamma}} = 2R$$
\begin{center}
\begin{tikzpicture}[line cap=round,line join=round,>=triangle 45,x=1cm,y=1cm, every node/.style={scale=2}]
\draw [shift={(5.548711354674366,8.366382786177725)},line width=2pt,color=atfczz,fill=atfczz,fill opacity=0.1] (0,0) -- (-119.13591063400457:0.5866114561766737) arc (-119.13591063400457:-54.25843155483686:0.5866114561766737) -- cycle;
\draw [shift={(8.840482833135399,3.792404346824896)},line width=2pt,color=atfczz,fill=atfczz,fill opacity=0.1] (0,0) -- (125.74156844516317:0.5866114561766737) arc (125.74156844516317:194.6830501714329:0.5866114561766737) -- cycle;
\draw [shift={(2,2)},line width=2pt,color=atfczz,fill=atfczz,fill opacity=0.1] (0,0) -- (14.68305017143291:0.5866114561766737) arc (14.68305017143291:60.864089365995454:0.5866114561766737) -- cycle;
\draw [line width=2pt] (5,4.5) circle (3.905124837953328cm);
\draw [line width=2pt] (5.548711354674366,8.366382786177725)-- (2,2);
\draw [line width=2pt] (2,2)-- (8.840482833135399,3.792404346824896);
\draw [line width=2pt] (8.840482833135399,3.792404346824896)-- (5.548711354674366,8.366382786177725);
\draw [line width=2pt] (5,4.5)-- (5.548711354674366,8.366382786177725);
\begin{scriptsize}
\draw [fill=qqqqff] (5,4.5) circle (2.5pt);
\draw [fill=qqqqff] (2,2) circle (2.5pt);
\draw [fill=qqqqff] (8.840482833135399,3.792404346824896) circle (2.5pt);
\draw [fill=qqqqff] (5.548711354674366,8.366382786177725) circle (2.5pt);
\draw[color=black] (3.4582470669427163,5.606050149528411) node {$b$};
\draw[color=black] (5.581780538302275,2.6) node {$a$};
\draw[color=black] (7.412008281573498,6.25) node {$c$};
\draw[color=black] (5.5465838509316745,6.14573268921095) node {$R$};
\draw[color=atfczz] (5.78,7.365884518058432) node {$\alpha$};
\draw[color=atfczz] (7.9516908212560375,4.033931446974926) node {$\beta$};
\draw[color=atfczz] (2.977225672877844,2.7316540142627104) node {$\gamma$};
\end{scriptsize}
\end{tikzpicture}
\end{center}
}
\block{Теорема косинусов}
{
$$c^2 = a^2 + b^2 - 2ab\cdot\cos{\gamma}$$
\begin{center}
\begin{tikzpicture}[line cap=round,line join=round,>=triangle 45,x=1cm,y=1cm, every node/.style={scale=2}]
\draw [shift={(8,7)},line width=2pt,color=atfczz,fill=atfczz,fill opacity=0.1] (0,0) -- (-138.81407483429035:0.5798090040927694) arc (-138.81407483429035:-60.25511870305777:0.5798090040927694) -- cycle;
\draw [line width=2pt] (0,0)-- (8,7);
\draw [line width=2pt] (8,7)-- (12,0);
\draw [line width=2pt] (12,0)-- (0,0);
\begin{scriptsize}
\draw [fill=qqqqff] (0,0) circle (2.5pt);
\draw [fill=qqqqff] (12,0) circle (2.5pt);
\draw [fill=qqqqff] (8,7) circle (2.5pt);
\draw[color=black] (3.638472032742155,4.079922692132785) node {$a$};
\draw[color=black] (10.422237380627555,3.859595270577533) node {$b$};
\draw[color=black] (6.154843110504774,-0.4) node {$c$};
\draw[color=atfczz] (7.8,5.8) node {$\gamma$};
\end{scriptsize}
\end{tikzpicture}
\end{center}
}
\block{Теорема Менелая}
{
$$\frac {AN} {NB} \cdot \frac {BM} {MC} \cdot \frac {CK} {KA} = 1$$
\begin{center}
\begin{tikzpicture}[line cap=round,line join=round,>=triangle 45,x=1cm,y=1cm, every node/.style={scale=2}]
\draw [line width=2pt] (0,0)-- (7.336438512897651,0);
\draw [line width=2pt,color=atfczz] (7.336438512897651,0)-- (12,0);
\draw [line width=2pt] (12,0)-- (6.12821251950331,3.284264399432161);
\draw [line width=2pt,color=atfczz] (6.12821251950331,3.284264399432161)-- (4.727510679155867,7.091726921860105);
\draw [line width=2pt] (4.727510679155867,7.091726921860105)-- (3.2591369739470513,4.8890232066693);
\draw [line width=2pt,color=atfczz] (3.2591369739470513,4.8890232066693)-- (0,0);
\draw [line width=2pt] (6.12821251950331,3.284264399432161)-- (7.336438512897651,0);
\draw [line width=2pt] (6.12821251950331,3.284264399432161)-- (3.2591369739470513,4.8890232066693);
\begin{scriptsize}
\draw [fill=qqqqff] (0,0) circle (2.5pt);
\draw[color=qqqqff] (-0.6176097607053476,-0.41689464842111934) node {$A$};
\draw [fill=qqqqff] (12,0) circle (2.5pt);
\draw[color=qqqqff] (12.533083384984945,-0.18357589906209826) node {$K$};
\draw [fill=qqqqff] (4.727510679155867,7.091726921860105) circle (2.5pt);
\draw[color=qqqqff] (5.215358973270185,7.5) node {$B$};
\draw [fill=qqqqff] (7.336438512897651,0) circle (2.5pt);
\draw[color=qqqqff] (7.2515953313125525,-0.5017378300062179) node {$C$};
\draw [fill=qqqqff] (3.2591369739470513,4.8890232066693) circle (2.5pt);
\draw[color=qqqqff] (2.5427987533395773,5.4) node {$N$};
\draw [fill=qqqqff] (6.12821251950331,3.284264399432161) circle (2.5pt);
\draw[color=qqqqff] (6.509217492442939,4) node {$M$};
\end{scriptsize}
\end{tikzpicture}
\end{center}
}
\block{Теорема Чевы}
{
$$\frac {AM} {MB} \cdot \frac {BN} {NC} \cdot \frac {CK} {KA} = 1$$
\begin{center}
\begin{tikzpicture}[line cap=round,line join=round,>=triangle 45,x=1cm,y=1cm, every node/.style={scale=2}]
\draw [line width=2pt] (0,0)-- (5.097108797244822,0);
\draw [line width=2pt,color=atfczz] (5.097108797244822,0)-- (12,0);
\draw [line width=2pt] (12,0)-- (9.162989788286335,3.971814296399131);
\draw [line width=2pt,color=atfczz] (9.162989788286335,3.971814296399131)-- (7,7);
\draw [line width=2pt] (7,7)-- (3.443991599101451,3.443991599101451);
\draw [line width=2pt,color=atfczz] (3.443991599101451,3.443991599101451)-- (0,0);
\draw [line width=2pt] (0,0)-- (5.777942179977047,2.504522419476832);
\draw [line width=2pt] (5.777942179977047,2.504522419476832)-- (9.162989788286335,3.971814296399131);
\draw [line width=2pt] (3.443991599101451,3.443991599101451)-- (5.777942179977047,2.504522419476832);
\draw [line width=2pt] (5.777942179977047,2.504522419476832)-- (12,0);
\draw [line width=2pt] (5.097108797244822,0)-- (5.777942179977047,2.504522419476832);
\draw [line width=2pt] (5.777942179977047,2.504522419476832)-- (7,7);
\begin{scriptsize}
\draw [fill=qqqqff] (0,0) circle (2.5pt);
\draw[color=qqqqff] (-0.39437262253997724,-0.3) node {$A$};
\draw [fill=qqqqff] (12,0) circle (2.5pt);
\draw[color=qqqqff] (12.360450670670039,-0.3) node {$C$};
\draw [fill=qqqqff] (7,7) circle (2.5pt);
\draw[color=qqqqff] (7.188718837851051,7.472726317656461) node {$B$};
\draw [fill=qqqqff] (3.4439915991014507,3.4439915991014507) circle (2.5pt);
\draw[color=qqqqff] (2.82606161691479,3.8239584601461334) node {$M$};
\draw [fill=qqqqff] (9.162989788286335,3.9718142963991308) circle (2.5pt);
\draw[color=qqqqff] (9.584214257346964,4.315748910506222) node {$N$};
\draw [fill=qqqqff] (5.777942179977047,2.504522419476832) circle (2.5pt);
\draw [fill=qqqqff] (5.097108797244822,0) circle (2.5pt);
\draw[color=qqqqff] (4.967729707192591,-0.5) node {$K$};
\end{scriptsize}
\end{tikzpicture}
\end{center}
}
\block{Теорема Ван-Обеля}
{
$$\frac {BO} {OK} = \frac {BN} {NC} + \frac {BM} {MA}$$
\begin{center}
\input{figures/van-obel_plain.tex}
\end{center}
}

\column{0.166}
\block{Свойство биссектрисы треугольника}
{
$$\frac {a} {b} = \frac {x} {y}$$
\begin{center}
\begin{tikzpicture}[line cap=round,line join=round,>=triangle 45,x=1cm,y=1cm, every node/.style={scale=2}]
\draw [shift={(0,0)},line width=2pt,color=atfczz,fill=atfczz,fill opacity=0.1] (0,0) -- (28.15496623701011:1.2926624362985586) arc (28.15496623701011:56.309932474020215:1.2926624362985586) -- cycle;
\draw [shift={(0,0)},line width=2pt,color=atfczz,fill=atfczz,fill opacity=0.1] (0,0) -- (0:1.6158280453731981) arc (0:28.15496623701011:1.6158280453731981) -- cycle;
\draw [line width=2pt] (0,0)-- (4,6);
\draw [line width=2pt] (4,6)-- (7.002889618359631,3.7478327862302785);
\draw [line width=2pt] (7.002889618359631,3.7478327862302785)-- (12,0);
\draw [line width=2pt] (0,0)-- (12,0);
\draw [line width=2pt] (0,0)-- (7.002889618359631,3.7478327862302785);
\begin{scriptsize}
\draw [fill=qqqqff] (0,0) circle (2.5pt);
\draw [fill=qqqqff] (12,0) circle (2.5pt);
\draw [fill=qqqqff] (4,6) circle (2.5pt);
\draw [fill=qqqqff] (7.002889618359631,3.7478327862302785) circle (2.5pt);
\draw[color=black] (1.3487898067234787,3.4636419534641294) node {$a$};
\draw[color=black] (5.856950053314702,5.370319047004501) node {$x$};
\draw[color=black] (10.025786410377552,2.2356126389805007) node {$y$};
\draw[color=black] (5.84079177286097,-0.36587051407034527) node {$b$};
\end{scriptsize}
\end{tikzpicture}
\end{center}
}
\block{Площадь треугольника}
{
$$S_\Delta = \frac 1 2 \cdot ah = \frac 1 2 \cdot ab \cdot \sin{\alpha}$$
\begin{center}
\input{figures/triangle_area_plain.tex}
\end{center}
}
\block{Формула Герона}
{
$$p = \frac {a + b + c} {2},$$
$$S_\Delta = \sqrt{p(p-a)(p-b)(p-c)}$$
\vspace{0.5cm}
\begin{center}
\begin{tikzpicture}[line cap=round,line join=round,>=triangle 45,x=1cm,y=1cm, every node/.style={scale=2}]
\draw [line width=2pt] (0,0)-- (4,5);
\draw [line width=2pt] (4,5)-- (12,0);
\draw [line width=2pt] (12,0)-- (0,0);
\begin{scriptsize}
\draw [fill=qqqqff] (0,0) circle (2.5pt);
\draw [fill=qqqqff] (12,0) circle (2.5pt);
\draw [fill=qqqqff] (4,5) circle (2.5pt);
\draw[color=black] (1.4625780189959279,2.545617367706919) node {$a$};
\draw[color=black] (8.426234735413837,2.6851696065128894) node {$b$};
\draw[color=black] (5.844518317503391,-0.4) node {$c$};
\end{scriptsize}
\end{tikzpicture}
\end{center}
}
\block{Площадь трапеции}
{
$$S_{ABCD} = \frac {AD + BC} {2}\cdot h$$
\begin{center}
\begin{tikzpicture}[line cap=round,line join=round,>=triangle 45,x=1cm,y=1cm, every node/.style={scale=2}]
\draw [line width=2pt] (0,0)-- (4,6);
\draw [line width=2pt] (4,6)-- (10,6);
\draw [line width=2pt] (10,6)-- (12,0);
\draw [line width=2pt] (12,0)-- (0,0);
\draw [line width=2pt] (4,6)-- (4,0);
\begin{scriptsize}
\draw [fill=ududff] (0,0) circle (2.5pt);
\draw[color=ududff] (-0.517884047721097,0.0063853552785903545) node {$A$};
\draw [fill=ududff] (12,0) circle (2.5pt);
\draw[color=ududff] (12.521769933234086,-0.06353236043698701) node {$D$};
\draw [fill=ududff] (4,6) circle (2.5pt);
\draw[color=ududff] (3.5723023216401804,6.491253487898394) node {$B$};
\draw [fill=ududff] (10,6) circle (2.5pt);
\draw[color=ududff] (10.231964743548925,6.508732916827288) node {$C$};
\draw [fill=qqqqff] (4,0) circle (2.5pt);
\draw[color=black] (4.673506344160524,3.082764846763996) node {$h$};
\end{scriptsize}
\end{tikzpicture}
\end{center}
}
\block{Площадь четырёхугольника}
{
$$S_{ABCD} = \frac 1 2 \cdot AC \cdot BD \cdot \sin{\alpha}$$
\begin{center}
\begin{tikzpicture}[line cap=round,line join=round,>=triangle 45,x=1cm,y=1cm, every node/.style={scale=2}]
\draw [shift={(4.08185340810201,3.724680814581253)},line width=2pt,color=atfczz,fill=atfczz,fill opacity=0.1] (0,0) -- (-30.421455835846963:1.2215700440189525) arc (-30.421455835846963:42.38037139310335:1.2215700440189525) -- cycle;
\draw [line width=2pt] (0,0)-- (7.134803867741345,6.5104903153131986);
\draw [line width=2pt] (2.1766666302526554,4.843406490534393)-- (11.158799306862601,-0.4309018171709668);
\draw [line width=2pt] (0,0)-- (2.1766666302526554,4.843406490534393);
\draw [line width=2pt] (2.1766666302526554,4.843406490534393)-- (7.134803867741345,6.5104903153131986);
\draw [line width=2pt] (7.134803867741345,6.5104903153131986)-- (11.158799306862601,-0.4309018171709668);
\draw [line width=2pt] (11.158799306862601,-0.4309018171709668)-- (0,0);
\begin{scriptsize}
\draw [fill=ududff] (0,0) circle (2.5pt);
\draw[color=ududff] (-0.3814447560458568,-0.21533063293232813) node {$A$};
\draw [fill=ududff] (11.158799306862601,-0.4309018171709668) circle (2.5pt);
\draw[color=ududff] (11.518084613926998,-0.6177301768444536) node {$D$};
\draw [fill=ududff] (2.1766666302526554,4.843406490534393) circle (2.5pt);
\draw[color=ududff] (1.616181551232195,5.217063209881367) node {$B$};
\draw [fill=ududff] (7.134803867741345,6.5104903153131986) circle (2.5pt);
\draw[color=ududff] (7.450974937958016,6.956004096073053) node {$C$};
\draw [fill=qqqqff] (4.08185340810201,3.724680814581253) circle (2.5pt);
\draw[color=atfczz] (6,3.9092646921669587) node {$\alpha$};
\end{scriptsize}
\end{tikzpicture}
\end{center}
}

\column{0.166}
\block{Отношение площадей треугольников с общим углом}
{
$$\frac {S_{\Delta ABC}} {S_{\Delta ADE}} = \frac {AB} {AD} \cdot \frac {AC} {AE}$$
\begin{center}
\begin{tikzpicture}[line cap=round,line join=round,>=triangle 45,x=1cm,y=1cm, every node/.style={scale=2}]
\draw [line width=2pt] (0,0)-- (8.886509083400261,0);
\draw [line width=2pt] (8.886509083400261,0)-- (12,0);
\draw [line width=2pt] (0,0)-- (3.9927778967520524,2.4954861854700328);
\draw [line width=2pt] (3.9927778967520524,2.4954861854700328)-- (8,5);
\draw [line width=2pt] (8,5)-- (12,0);
\draw [line width=2pt] (8.886509083400261,0)-- (3.9927778967520524,2.4954861854700328);
\begin{scriptsize}
\draw [fill=qqqqff] (0,0) circle (2.5pt);
\draw[color=qqqqff] (-0.6724863111319213,-0.17184458355641336) node {$A$};
\draw [fill=qqqqff] (12,0) circle (2.5pt);
\draw[color=qqqqff] (12.496152770605136,-0.2589099659150054) node {$C$};
\draw [fill=qqqqff] (8,5) circle (2.5pt);
\draw[color=qqqqff] (8.208182689444474,5.5962369977003075) node {$B$};
\draw [fill=qqqqff] (8.886509083400261,0) circle (2.5pt);
\draw[color=qqqqff] (8.861173057133914,-0.4112743850425414) node {$E$};
\draw [fill=qqqqff] (3.9927778967520524,2.4954861854700328) circle (2.5pt);
\draw[color=qqqqff] (3.39782031413226,3.1584062916597313) node {$D$};
\end{scriptsize}
\end{tikzpicture}
\end{center}
}
\block{Углы, опирающиеся на дугу}
{
$$\angle{AEB} = \angle{ADB} = \frac 1 2 \smallsmile{AB}$$
\begin{center}
\begin{tikzpicture}[line cap=round,line join=round,>=triangle 45,x=1cm,y=1cm, every node/.style={scale=2}]
\draw [shift={(1.6520687866545791,6.510312560451101)},line width=2pt,color=atfczz,fill=atfczz,fill opacity=0.1] (0,0) -- (-41.29625435879208:0.8524939824945308) arc (-41.29625435879208:0.03759287605244468:0.8524939824945308) -- cycle;
\draw [shift={(1.2248415885962336,3.5009109304934847)},line width=2pt,color=atfczz,fill=atfczz,fill opacity=0.1] (0,0) -- (-18.392904592537125:0.8524939824945308) arc (-18.392904592537125:22.940942642307405:0.8524939824945308) -- cycle;
\draw [line width=2pt] (5,4.5) circle (3.905124837953328cm);
\draw [line width=2pt] (1.6520687866545791,6.510312560451101)-- (8.345290321189832,6.514704113996304);
\draw [line width=2pt] (1.6520687866545791,6.510312560451101)-- (7.425168005664267,1.4391896262096853);
\draw [line width=2pt] (1.2248415885962336,3.5009109304934847)-- (8.345290321189832,6.514704113996304);
\draw [line width=2pt] (1.2248415885962336,3.5009109304934847)-- (7.425168005664267,1.4391896262096853);
\draw [shift={(5,4.5)},line width=2pt,color=atfczz]  plot[domain=-0.9007505215650342:0.5420729319637477,variable=\t]({1*3.9051248379533274*cos(\t r)+0*3.9051248379533274*sin(\t r)},{0*3.9051248379533274*cos(\t r)+1*3.9051248379533274*sin(\t r)});
\begin{scriptsize}
\draw [fill=qqqqff] (8.345290321189832,6.514704113996304) circle (2.5pt);
\draw[color=qqqqff] (8.689192846598528,6.80966037199125) node {$A$};
\draw [fill=qqqqff] (7.425168005664267,1.4391896262096853) circle (2.5pt);
\draw[color=qqqqff] (7.768499345504433,1.3) node {$B$};
\draw [fill=xdxdff] (1.2248415885962336,3.5009109304934847) circle (2.5pt);
\draw[color=xdxdff] (0.8007818619158017,3.308751750547044) node {$D$};
\draw [fill=qqqqff] (1.6520687866545791,6.510312560451101) circle (2.5pt);
\draw[color=qqqqff] (1.1872458006466557,6.775560612691469) node {$E$};
\end{scriptsize}
\end{tikzpicture}
\end{center}
}
\block{Угол между касательной и хордой}
{
$$\angle{ABC} = \angle{ADB} = \frac 1 2 \smallsmile{AB}$$
\begin{center}
\begin{tikzpicture}[line cap=round,line join=round,>=triangle 45,x=1cm,y=1cm, every node/.style={scale=2}]
\draw [shift={(1.8488398961733041,2.1934462936989885)},line width=2pt,color=atfczz,fill=atfczz,fill opacity=0.1] (0,0) -- (74.01519770755257:1.0033149248000326) arc (74.01519770755257:126.20299721316648:1.0033149248000326) -- cycle;
\draw [shift={(8.151160103826696,6.8065537063010115)},line width=2pt,color=atfczz,fill=atfczz,fill opacity=0.1] (0,0) -- (164.01519770755257:1.0033149248000326) arc (164.01519770755257:216.20299721316647:1.0033149248000326) -- cycle;
\draw[line width=2pt,color=atfczz,fill=atfczz,fill opacity=0.1] (2.110737148194265,1.8356483505529824) -- (2.4685350913402715,2.0975456025739434) -- (2.2066378393193102,2.4553435457199497) -- (1.8488398961733041,2.1934462936989885) -- cycle; 
\draw [line width=2pt] (5,4.5) circle (3.905124837953328cm);
\draw [line width=2pt] (3.548028564805617,8.125159162213373)-- (8.151160103826696,6.8065537063010115);
\draw [line width=2pt] (8.151160103826696,6.8065537063010115)-- (1.8488398961733041,2.1934462936989885);
\draw [line width=2pt] (1.8488398961733041,2.1934462936989885)-- (3.548028564805617,8.125159162213373);
\draw [line width=2pt] (-1.223135101827636,6.390307189332673)-- (2.939181834097665,0.7038464613038675);
\draw [shift={(5,4.5)},line width=2pt,color=atfczz]  plot[domain=1.9517562925626166:3.7734541540498796,variable=\t]({1*3.9051248379533283*cos(\t r)+0*3.9051248379533283*sin(\t r)},{0*3.9051248379533283*cos(\t r)+1*3.9051248379533283*sin(\t r)});
\begin{scriptsize}
\draw [fill=qqqqff] (1.8488398961733041,2.1934462936989885) circle (2.5pt);
\draw[color=qqqqff] (1.4209984726626994,2.0) node {$B$};
\draw [fill=qqqqff] (8.151160103826696,6.8065537063010115) circle (2pt);
\draw[color=qqqqff] (8.5,7) node {$D$};
\draw [fill=xdxdff] (3.548028564805617,8.125159162213373) circle (2.5pt);
\draw[color=xdxdff] (2.9,8.4) node {$A$};
\draw [fill=xdxdff] (-0.7537487755685042,5.7490424918973) circle (2.5pt);
\draw[color=xdxdff] (-1.1625374586973845,5.5) node {$C$};
\end{scriptsize}
\end{tikzpicture}
\end{center}
}
\block{Угол между хордами}
{
$$\angle{AKB} = \angle{EKD} = \frac {\smallsmile{AB} \ + \smallsmile{ED}} {2}$$
\begin{center}
\input{figures/chord_angle_plain.tex}
\end{center}
}
\block{Угол между секущими}
{
$$\angle{BAC} = \frac {\smallsmile {DE} \ - \smallsmile{BC}} {2}$$
\begin{center}
\input{figures/sec_angle_plain.tex}
\end{center}
}

\column{0.166}
\block{Углы при параллельных прямых}
{
$$k \ \parallelsum \ l \iff \alpha = \beta = \gamma$$
\begin{center}
\input{figures/parallel_plain.tex}
\end{center}
}
\block{Касательные, проведённые из одной точки}
{
$$AB = AC$$
\begin{center}
\input{figures/2kas_plain.tex}
\end{center}
}
\block{О касательной и секущей}
{
$$\Delta ABC \sim \Delta ADB, \quad AB^2 = AC \cdot AD$$
\begin{center}
\input{figures/kas_plain.tex}
\end{center}
}
\block{О пересекающихся хордах}
{
$$\Delta AKB \sim \Delta EKD, \quad AK \cdot KD = BK \cdot KE$$
\vspace{0.5cm}
\begin{center}
\input{figures/chord_plain.tex}
\end{center}
}
\block{О двух секущих}
{
$$\Delta ABC \sim \Delta ADE, \quad AB \cdot AD = AC \cdot AE$$
\begin{center}
\input{figures/sec_plain.tex}
\end{center}
}

\column{0.166}
\block{Вписанная окружность треугольника}
{
\begin{center}
Центр вписанной окружности - точка пересечения биссектрис треугольника \\
\vspace{0.5cm}
\begin{tikzpicture}[line cap=round,line join=round,>=triangle 45,x=1cm,y=1cm, every node/.style={scale=2}]
\draw [shift={(0,0)},line width=2pt,color=atfczz,fill=atfczz,fill opacity=0.1] (0,0) -- (0:1.3589449118673937) arc (0:19.750539888170113:1.3589449118673937) -- cycle;
\draw [shift={(0,0)},line width=2pt,color=atfczz,fill=atfczz,fill opacity=0.1] (0,0) -- (19.750539888170113:1.509938790963771) arc (19.750539888170113:39.501079776340234:1.509938790963771) -- cycle;
\draw [shift={(12,0)},line width=2pt,color=duqsxz,fill=duqsxz,fill opacity=0.1] (0,0) -- (147.56518573968398:1.1324540932228282) arc (147.56518573968398:180:1.1324540932228282) -- cycle;
\draw [shift={(12,0)},line width=2pt,color=duqsxz,fill=duqsxz,fill opacity=0.1] (0,0) -- (115.130371479368:1.2834479723192054) arc (115.130371479368:147.56518573968398:1.2834479723192054) -- cycle;
\draw [shift={(8.653668027821256,7.133807334953998)},line width=2pt,fill=black,fill opacity=0.1] (0,0) -- (-102.6842743721459:0.9059632745782625) arc (-102.6842743721459:-64.86962852063203:0.9059632745782625) -- cycle;
\draw [shift={(8.653668027821256,7.133807334953998)},line width=2pt,fill=black,fill opacity=0.1] (0,0) -- (-140.49892022365978:1.0569571536746396) arc (-140.49892022365978:-102.6842743721459:1.0569571536746396) -- cycle;
\draw [line width=2pt] (7.667687961755012,2.7530628447828813) circle (2.7530628447828827cm);
\draw [line width=2pt] (0,0)-- (7.667687961755012,2.7530628447828813);
\draw [line width=2pt] (7.667687961755012,2.7530628447828813)-- (12,0);
\draw [line width=2pt] (7.667687961755012,2.7530628447828813)-- (8.653668027821256,7.133807334953998);
\draw [line width=2pt] (0,0)-- (5.91648461304076,4.877360813236461);
\draw [line width=2pt] (5.91648461304076,4.877360813236461)-- (8.653668027821256,7.133807334953998);
\draw [line width=2pt] (8.653668027821256,7.133807334953998)-- (10.160156370078592,3.922231891941126);
\draw [line width=2pt] (10.160156370078592,3.922231891941126)-- (12,0);
\draw [line width=2pt] (12,0)-- (7.667687961755013,0);
\draw [line width=2pt] (7.667687961755013,0)-- (0,0);
\begin{scriptsize}
\draw [fill=ududff] (0,0) circle (2.5pt);
\draw [fill=ududff] (12,0) circle (2.5pt);
\draw [fill=ududff] (8.653668027821256,7.133807334953998) circle (2.5pt);
\draw [fill=qqqqff] (7.667687961755012,2.7530628447828813) circle (2.5pt);
\draw[color=qqqqff] (7.581611486236979,2.2718044280506553) node {$O$};
\draw [fill=qqqqff] (7.667687961755013,0) circle (2.5pt);
\draw [fill=qqqqff] (10.160156370078592,3.922231891941126) circle (2.5pt);
\draw [fill=qqqqff] (5.91648461304076,4.877360813236461) circle (2.5pt);
\end{scriptsize}
\end{tikzpicture}
\end{center}
}
\block{Описанная окружность треугольника}
{
\begin{center}
Центр описанной окружности - точка пересечения серединных перпендикуляров к сторонам треугольника \\
\vspace{0.5cm}
\begin{tikzpicture}[line cap=round,line join=round,>=triangle 45,x=1cm,y=1cm, every node/.style={scale=2}]
\draw[line width=2pt,color=atfczz,fill=atfczz,fill opacity=0.1] (3.3736544181191737,5.007787331059846) -- (3.4972791211804592,5.403733581265436) -- (3.1013328709748693,5.527358284326721) -- (2.9777081679135837,5.131412034121131) -- cycle; 
\draw[line width=2pt,color=atfczz,fill=atfczz,fill opacity=0.1] (5.001385724320344,2.416793516647576) -- (4.5865888775047745,2.4165175986897416) -- (4.586864795462609,2.001720751874172) -- (5.0016616422781786,2.0019966698320064) -- cycle; 
\draw[line width=2pt,color=atfczz,fill=atfczz,fill opacity=0.1] (5.631069835631049,4.908145240471232) -- (5.856333299112954,4.559845265910517) -- (6.20463327367367,4.785108729392423) -- (5.979369810191764,5.133408703953138) -- cycle; 
\draw [line width=2pt] (5,4.5) circle (3.905124837953328cm);
\draw [line width=2pt] (2.9777081679135837,5.131412034121131)-- (3.9554163358271683,8.262824068242264);
\draw [line width=2pt] (3.399367945028995,6.718097858987331) -- (3.533756558711758,6.676138243376063);
\draw [line width=2pt,color=atfczz] (3.9554163358271683,8.262824068242264)-- (5.979369810191764,5.133408703953138);
\draw [line width=2pt,color=atfczz] (4.994644474228897,6.785601918297088) -- (4.876427542757385,6.709144963457908);
\draw [line width=2pt,color=atfczz] (5.026501538745222,6.7363448635172904) -- (4.9082846072737105,6.659887908678111);
\draw [line width=2pt,color=atfczz] (5.0583586032615475,6.687087808737492) -- (4.940141671790036,6.610630853898313);
\draw [line width=2pt] (5.979369810191764,5.133408703953138)-- (5,4.5);
\draw [line width=2pt] (5,4.5)-- (5.0016616422781786,2.0019966698320064);
\draw [line width=2pt,color=duqsxz] (5.0016616422781786,2.0019966698320064)-- (8.003323284556359,2.0039933396640133);
\draw [line width=2pt,color=duqsxz] (6.4731150722672925,2.0733688535696477) -- (6.473208721927629,1.9325821352345656);
\draw [line width=2pt,color=duqsxz] (6.53177620490691,2.0734078742614535) -- (6.531869854567246,1.9326211559263717);
\draw [line width=2pt,color=atfczz] (8.003323284556359,2.0039933396640133)-- (5.979369810191764,5.133408703953138);
\draw [line width=2pt,color=atfczz] (6.964095146154629,3.4812154896091894) -- (7.082312077626143,3.5576724444483685);
\draw [line width=2pt,color=atfczz] (6.932238081638305,3.5304725443889855) -- (7.0504550131098185,3.6069294992281646);
\draw [line width=2pt,color=atfczz] (6.900381017121981,3.5797295991687816) -- (7.018597948593493,3.6561865540079608);
\draw [line width=2pt] (5,4.5)-- (2.9777081679135837,5.131412034121131);
\draw [line width=2pt] (2.9777081679135837,5.131412034121131)-- (2,2);
\draw [line width=2pt] (2.556048390798174,3.544726209254932) -- (2.4216597771154103,3.5866858248662);
\draw [line width=2pt,color=duqsxz] (2,2)-- (5.0016616422781786,2.0019966698320064);
\draw [line width=2pt,color=duqsxz] (3.4714534299891118,2.0713721837376418) -- (3.4715470796494494,1.93058546540256);
\draw [line width=2pt,color=duqsxz] (3.530114562628729,2.071411204429448) -- (3.5302082122890663,1.930624486094366);
\begin{scriptsize}
\draw [fill=qqqqff] (5,4.5) circle (2.5pt);
\draw[color=qqqqff] (4.58454106280193,4.2) node {$O$};
\draw [fill=qqqqff] (2,2) circle (2.5pt);
\draw [fill=qqqqff] (8.003323284556359,2.0039933396640133) circle (2.5pt);
\draw [fill=qqqqff] (3.9554163358271683,8.262824068242264) circle (2.5pt);
\draw [fill=qqqqff] (2.9777081679135837,5.131412034121131) circle (2.5pt);
\draw [fill=qqqqff] (5.0016616422781786,2.0019966698320064) circle (2.5pt);
\draw [fill=qqqqff] (5.979369810191764,5.133408703953138) circle (2.5pt);
\end{scriptsize}
\end{tikzpicture}
\end{center}
}
\block{Описанная окружность прямоугольного треугольника}
{
\begin{center}
$\Delta ABC$ - прямоугольный $\iff$ $AC$ - диаметр,\\
AO = OC = OB = R\\
\vspace{0.5cm}
\begin{tikzpicture}[line cap=round,line join=round,>=triangle 45,x=1cm,y=1cm, every node/.style={scale=2}]
\draw[line width=2pt,color=atfczz,fill=atfczz,fill opacity=0.1] (6.5955674367606445,7.673331964049699) -- (6.802843706528104,7.31403675360271) -- (7.162138916975093,7.521313023370169) -- (6.954862647207634,7.880608233817158) -- cycle; 
\draw [shift={(5,4.5)},line width=2pt]  plot[domain=-0.27096542946848157:3.440696442705099,variable=\t]({1*3.9051248379533283*cos(\t r)+0*3.9051248379533283*sin(\t r)},{0*3.9051248379533283*cos(\t r)+1*3.9051248379533283*sin(\t r)});
\draw [line width=2pt,color=duqsxz] (1.094875162046672,4.5)-- (5,4.5);
\draw [line width=2pt,color=duqsxz] (3.047437581023336,4.570393374741201) -- (3.047437581023336,4.429606625258799);
\draw [line width=2pt,color=duqsxz] (5,4.5)-- (8.905124837953329,4.5);
\draw [line width=2pt,color=duqsxz] (6.952562418976664,4.570393374741201) -- (6.952562418976664,4.429606625258799);
\draw [line width=2pt,color=duqsxz] (5,4.5)-- (6.954862647207634,7.880608233817158);
\draw [line width=2pt,color=duqsxz] (5.916492830692282,6.225542267954368) -- (6.038369816515353,6.15506596586279);
\draw [line width=2pt] (1.094875162046672,4.5)-- (6.954862647207634,7.880608233817158);
\draw [line width=2pt] (6.954862647207634,7.880608233817158)-- (8.905124837953329,4.5);
\begin{scriptsize}
\draw [fill=qqqqff] (5,4.5) circle (2.5pt);
\draw[color=qqqqff] (4.877846790890267,4.186450425580861) node {$O$};
\draw [fill=qqqqff] (1.094875162046672,4.5) circle (2.5pt);
\draw[color=qqqqff] (0.5838509316770155,4.620542903151599) node {$A$};
\draw [fill=qqqqff] (8.905124837953329,4.5) circle (2.5pt);
\draw[color=qqqqff] (9.312629399585921,4.491488382792731) node {$C$};
\draw [fill=xdxdff] (6.954862647207634,7.880608233817158) circle (2.5pt);
\draw[color=xdxdff] (7.142167011732227,8.23406947319991) node {$B$};
\end{scriptsize}
\end{tikzpicture}
\end{center}
}
\block{Вписанный четырёхугольник}
{
\begin{center}
$ABCD$ - вписанный $\iff \alpha + \beta = 180^\circ$\\
\vspace{0.5cm}
\input{figures/v4_plain.tex}
\end{center}
}
\block{Описанный четырёхугольник}
{
\begin{center}
$ABCD$ - описанный $\iff AB + CD = BC + AD$\\
\vspace{0.5cm}
\begin{tikzpicture}[line cap=round,line join=round,>=triangle 45,x=1cm,y=1cm, every node/.style={scale=2}]
\draw [line width=2pt] (5,4.5) circle (3.905124837953328cm);
\draw [line width=2pt,color=atfczz] (0.9174339898196684,8.181623271591448)-- (4.780652856691825,8.398959711348983);
\draw [line width=2pt,color=atfczz] (2.8450894944319516,8.360573734279559) -- (2.8529973520795426,8.22000924866087);
\draw [line width=2pt,color=duqsxz] (4.780652856691825,8.398959711348983)-- (9.342418674061229,8.655594910109388);
\draw [line width=2pt,color=duqsxz] (7.028297568715504,8.59591208319528) -- (7.036205426363096,8.455347597576594);
\draw [line width=2pt,color=duqsxz] (7.086866104389957,8.599207023881776) -- (7.094773962037548,8.45864253826309);
\draw [line width=2pt,color=duqsxz] (9.342418674061229,8.655594910109388)-- (8.885552914823354,4.109514986065079);
\draw [line width=2pt,color=duqsxz] (9.186959218690427,6.404699677985135) -- (9.046878072115213,6.418777362595673);
\draw [line width=2pt,color=duqsxz] (9.18109351676937,6.346332533578793) -- (9.041012370194155,6.360410218189332);
\draw [line width=2pt,color=atfczz] (8.885552914823354,4.109514986065079)-- (8.510505082928951,0.37757100765069546);
\draw [line width=2pt,color=atfczz] (8.773935274084819,2.2948712989589577) -- (8.633854127509604,2.3089489835694965);
\draw [line width=2pt,color=atfczz] (8.76806957216376,2.236504154552618) -- (8.627988425588546,2.2505818391631567);
\draw [line width=2pt,color=atfczz] (8.762203870242702,2.178137010146278) -- (8.622122723667488,2.192214694756817);
\draw [line width=2pt,color=atfczz] (8.510505082928951,0.37757100765069546)-- (4.766475235597508,0.6018637550220531);
\draw [line width=2pt,color=atfczz] (6.6928368320224685,0.4159420712886857) -- (6.701255818320766,0.5564768694688149);
\draw [line width=2pt,color=atfczz] (6.634280666114082,0.41944998224630997) -- (6.64269965241238,0.5599847804264392);
\draw [line width=2pt,color=atfczz] (6.575724500205696,0.4229578932039342) -- (6.584143486503993,0.5634926913840634);
\draw [line width=2pt,color=duqsxz] (4.766475235597508,0.6018637550220531)-- (1.2640003876462262,0.8116857218009754);
\draw [line width=2pt,color=duqsxz] (2.839569313896706,0.7172984712843857) -- (2.9337178024830095,0.8174377034830463);
\draw [line width=2pt,color=duqsxz] (2.839569313896706,0.7172984712843857) -- (2.921089323035564,0.6066355062128526);
\draw [line width=2pt,color=duqsxz] (3.0152378116218674,0.7067747384115144) -- (3.109386300208171,0.806913970610175);
\draw [line width=2pt,color=duqsxz] (3.0152378116218674,0.7067747384115144) -- (3.0967578207607245,0.5961117733399811);
\draw [line width=2pt,color=duqsxz] (1.2640003876462262,0.8116857218009754)-- (1.0991856921422158,4.316566808859458);
\draw [line width=2pt,color=duqsxz] (1.1733266707640744,2.7399154492636972) -- (1.2829333656892354,2.656980678775045);
\draw [line width=2pt,color=duqsxz] (1.1733266707640744,2.7399154492636972) -- (1.07198634496906,2.647061035818869);
\draw [line width=2pt,color=duqsxz] (1.1815930398942212,2.564126265330217) -- (1.2911997348193824,2.4811914948415645);
\draw [line width=2pt,color=duqsxz] (1.1815930398942212,2.564126265330217) -- (1.080252714099207,2.4712718518853887);
\draw [line width=2pt,color=atfczz] (1.0991856921422158,4.316566808859458)-- (0.9174339898196684,8.181623271591448);
\draw [line width=2pt,color=atfczz] (0.9379941674075506,6.245788492573395) -- (1.0786255145543346,6.252401587877511);
\begin{scriptsize}
\draw [fill=qqqqff] (8.885552914823354,4.109514986065079) circle (2.5pt);
\draw [fill=qqqqff] (4.780652856691825,8.398959711348983) circle (2.5pt);
\draw [fill=xdxdff] (4.766475235597508,0.6018637550220531) circle (2.5pt);
\draw [fill=xdxdff] (1.0991856921422158,4.316566808859458) circle (2.5pt);
\draw [fill=qqqqff] (0.9174339898196684,8.181623271591448) circle (2.5pt);
\draw[color=qqqqff] (0.5838509316770151,8.480446284794116) node {$B$};
\draw [fill=qqqqff] (9.342418674061229,8.655594910109388) circle (2.5pt);
\draw[color=qqqqff] (9.664596273291925,8.96146767885899) node {$C$};
\draw [fill=qqqqff] (8.510505082928951,0.37757100765069546) circle (2.5pt);
\draw[color=qqqqff] (8.890269151138716,0.3265470439383502) node {$D$};
\draw [fill=qqqqff] (1.2640003876462262,0.8116857218009754) circle (2.5pt);
\draw[color=qqqqff] (0.9006211180124188,0.7019783758914214) node {$A$};
\end{scriptsize}
\end{tikzpicture}
\end{center}
}

\end{columns}
\end{document}